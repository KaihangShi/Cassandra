\chapter{CBMC Widom Insertion Method}
\label{ch:widom_insertions}
The Widom insertion method, also known as the test particle insertion method, can be used to calculate the shifted chemical potential of a species during a simulation.  From shifted chemical potentials, excess chemical potentials and Henry's constants can be calculated. \\ \\
%
The chemical potential $\mu$ of a given species {\it a} in the NVT ensemble is defined in Equation \ref{eq:mu_nvt}, where {\it F} is the Helmholtz free energy.
\begin{equation}
    \label{eq:mu_nvt}
    \mu_a = {\left(\frac{\partial F}{\partial N_a}\right)}_{V,T,N_{b \neq a}}
\end{equation}

% \begin{equation}
%     \label{eq:mu_npt}
%     \mu_a = {\left(\frac{\partial G}{\partial N_a}\right)}_{P,T,N_{b \neq a}}
% \end{equation}
For the sake of simpler notation, the following derivation is for a pure component system, but it can be easily extended to multicomponent systems. \\ \\
%
The Helmholtz free energy is defined in Equation \ref{eq:Helmholtz}.
\begin{equation}
    \label{eq:Helmholtz}
    F(N,V,T) = -k_B T \ln{Q(N,V,T)}
\end{equation}
%
Approximating Equation \ref{eq:mu_nvt} with the addition of a single molecule and substituting Equation \ref{eq:Helmholtz} yields Equation \ref{eq:mu_approx}.
%
\begin{equation}
    \label{eq:mu_approx}
    \mu \approx F(N+1,V,T) - F(N,V,T) = -k_B T \ln{\left(\frac{Q(N+1,V,T)}{Q(N,V,T)}\right)}
\end{equation}
%
Equation \ref{eq:mu_approx} can be combined with Equation \ref{eq:partitionFn_NVT} to obtain Equation \ref{eq:mu_combination1}.
\begin{equation}
    \label{eq:mu_combination1}
    -\beta \mu = \ln{\left( \frac{Q(1,V,T) Z(N+1,V,T)}{(N+1) Z(1,V,T) Z(N,V,T)} \right)}
\end{equation}

The configurational partition function {\it Z(N,V,T)} is defined in Equation \ref{eq:Z_NVT}, which leads to \ref{eq:Z_insertion}, where $\Delta U = U(\mathbf{q}^{N+1}) - U(\mathbf{q}^N)$ is the change in potential energy of inserting the molecule, ${\langle ... \rangle}_N$ denotes NVT ensemble averaging over configurational space of the system of N particles, and $\mathbf{q}_{N+1}$ denotes the generalized coordinates of only the inserted molecule.

\begin{equation}
    \label{eq:Z_NVT}
    Z(N,V,T) = \int e^{-\beta U(\mathbf{q}^N)} d\mathbf{q}^N
\end{equation}

\begin{equation}
    \label{eq:Z_insertion}
    \frac{Z(N+1,V,T)}{Z(N,V,T)} =
    \frac{\int \left(\int e^{-\beta \Delta U}d\mathbf{q}_{N+1}\right) e^{-\beta U(\mathbf{q}^N)} d\mathbf{q}^N}{\int e^{-\beta U(\mathbf{q}^N)} d\mathbf{q}^N} =
    \left\langle{\int e^{-\beta \Delta U} d\mathbf{q}_{N+1}}\right\rangle_N
\end{equation}

The final integral in Equation \ref{eq:Z_insertion} can be estimated by CBMC importance sampling with test molecules inserted as described in Section \ref{sec:cbmcInsert}. This is demonstrated in Equation \ref{eq:Z_insertion_integral}, where $n_{IPC}$ is the number of Widom insertions into the configuration of N molecules. The overall probability $\alpha_{mn}$ of attempting the insertion with a given position, orientation, and conformation is defined by Equation \ref{eq:alpha_cbmcInsert}.   

\begin{equation}
    \label{eq:Z_insertion_integral}
    \int e^{-\beta \Delta U} d\mathbf{q}_{N+1} =
    \frac{1}{n_{IPC}} \sum_{i=1}^{n_{IPC}} {\left( {\frac{ e^{-\beta \Delta U}}{\alpha_{mn}} }\right)}_i
\end{equation}

When combined with Equation \ref{eq:Z_insertion_integral}, Equation \ref{eq:Z_insertion} can be be further transformed into a single arithmetic average (denoted by $\langle ... \rangle$ without a subscript) as in Equations \ref{eq:Z_insertion_doublesum} and \ref{eq:Z_insertion_avg}, where $n_{IPC}$ is the same for each N-molecule system configuration $j$ on which Widom insertions are performed, $n_{confs}$ is the total number of these system configurations, and $n_{ins}=n_{IPC} n_{confs}$ is the total number of Widom insertions.

\begin{align}
    \label{eq:Z_insertion_doublesum}
    \left\langle{\int e^{-\beta \Delta U} d\mathbf{q}_{N+1}}\right\rangle_N &= 
    \frac{1}{n_{confs} n_{IPC}} \sum_{j=1}^{n_{confs}} 
    \sum_{i=1}^{n_{IPC}} {\left( {\frac{ e^{-\beta \Delta U}}{\alpha_{mn}} }\right)}_{i,j}
    = \frac{1}{n_{ins}} \sum_{k=1}^{n_{ins}} {\left( {\frac{ e^{-\beta \Delta U}}{\alpha_{mn}} }\right)}_k \\
    \label{eq:Z_insertion_avg}
    &= V Z_{rot} Z_{frag} \Omega_{dih} \left\langle 
    \frac{\exp{(-\beta({\Delta}U-U_{frag} (\mathbf{q}_{frag,n}^{(N+1)} )))}}{p_{seq} p_{bias}}
    \right\rangle
\end{align}

Combining Equations \ref{eq:Z_insertion}, \ref{eq:Z_insertion_avg}, \ref{eq:partitionFn_1VT}, and \ref{eq:configPartitionFn_1VT} with Equation \ref{eq:mu_combination1} yields Equation \ref{eq:mu_general} for the chemical potential $\mu$.

\begin{equation}
    \label{eq:mu_general}
    \mu = -k_B T \ln{\left\langle \frac{V\Lambda^{-3}  \exp{(-\beta({\Delta}U-U (\mathbf{q}_{frag,n}^{(N+1)} )))}}{p_{seq} p_{bias} (N+1)} \right\rangle}
    - k_B T \ln\left( Q_{rot+int} \frac{Z_{frag}\Omega_{dih}}{Z_{int}} \right)
\end{equation}

The chemical potential $\mu$ cannot always be calculated in this way for relatively complex molecules, so the shifted chemical potential $\mu'$ defined in Equation \ref{eq:muShift} is calculated instead as in Equation \ref{eq:muShift_widom}, where \texttt{widom\_var} is a code variable defined in Equation \ref{eq:widom_var}.

\begin{equation}
    \label{eq:muShift_widom}
    \mu' = -k_B T \ln{\left\langle \texttt{widom\_var} \right\rangle}
\end{equation}


\begin{equation}
\label{eq:widom_var}
    \texttt{widom\_var} = \frac{V\Lambda^{-3}}{N+1} \exp{\left[-\beta({\Delta}U-U (\mathbf{q}_{frag,n}^{(N+1)} )) - \ln{(p_{seq} p_{bias})}\right]} 
\end{equation}

Unlike GCMC insertions, Widom insertions are never accepted and therefore do not have acceptance criteria.  The test molecule is only inserted for the sampling of \texttt{widom\_var} and is then always removed. While the derivation is different for other ensembles, the Widom insertion procedure and Equations \ref{eq:mu_general}, \ref{eq:muShift_widom}, and \ref{eq:widom_var} apply to all ensembles in Cassandra.  The identification of code variable names with symbols for \texttt{widom\_insert.f90} is the same as in Table \ref{table:cbmcInsert} for \texttt{move\_insert.f90}, except for $\mu'$, which is only calculated and written to the log file upon completion of the simulation.  As described in Section \ref{sec:output_files}, the average  \texttt{widom\_var} values for each system configuration (step) in which Widom insertions are performed are written to Widom property files.  The final value of $\mu'$ in kJ/mol for each test particle species in each box is written to the log file after the simulation is completed.
