{\huge \bf{Preface and Disclaimer}} \\ \\
%
Cassandra is free software: you can redistribute it and/or modify
it under the terms of the GNU General Public License as published by
the Free Software Foundation, either version 3 of the License, or
 (at your option) any later version. \\ \\
 %
 This program is distributed in the hope that it will be useful,
 but WITHOUT ANY WARRANTY; without even the implied warranty of
MERCHANTABILITY or FITNESS FOR A PARTICULAR PURPOSE.  See the
GNU General Public License for more details.  \\ \\
%
This user manual is distributed along with the Cassandra software to
aid in setting up various input files required for carrying out a Cassandra
Monte Carlo simulation. Every effort is made to release the most
updated and complete version of the manual when a new version of the
software is released. To report any inconsistencies, errors or missing
information, or to suggest improvements, send email to Edward Maginn
(ed@nd.edu). \\ \\  
%
\pagebreak
\\ \\
{\huge \bf{Acknowledgements}} \\ \\
Support for this work was provided by a grant from the National Science Foundation entitled ``SI2-SSE: Development of Cassandra, a General, Efficient and Parallel Monte Carlo Multiscale Modeling Software Platform for Materials Research'', grant number ACI-1339785 as well as grant  NSF OAC-1835630 "Collaborative Research: NSCI Framework: Software for Building a Community-Based Molecular Modeling Capability Around the Molecular Simulation Design Framework (MoSDeF)"		
\\ \\
Ed Maginn would like to acknowledge financial support from Sandia National Laboratory's Computer Science Research Institute, which enabled him to take a research leave and lay the foundation for Cassandra in collaboration with Jindal Shah, who stayed behind at Notre Dame and helped keep the group going. The hospitality of Steve Plimpton and co-workers at Sandia is gratefully acknowledged. 
\\ \\ 
Finally, we would also like to thank the Center for Research Computing at Notre Dame, which provided support, encouragement, and infrastructure to help bring Cassandra to life.
\\ \\
People who have contributed to Cassandra through algorithm development and / or writing code (to date) include:

\begin{itemize}

\item Ed Maginn
\item Jindal Shah
\item Eliseo Marin
\item Brian Keene
\item Sandip Khan
\item Ryan Gotchy Mullen
\item Andrew Paluch
\item Neeraj Rai
\item Lucienne Romanielo
\item Tom Rosch
\item Brian Yoo
\item Ryan DeFever
\item Ryan Smith
\end{itemize}


Some legacy code was used in the creation of Cassandra, and the following former students are recognized for their work:

\begin{itemize}
\item David Eike
\item Jim Larentzos
\item Craig Powers
\end{itemize}
