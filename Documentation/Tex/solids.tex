\chapter{Simulating Rigid Solids and Surfaces}

The simulations involving a rigid solid or surface can be done in the canonical and grand canonical ensemble
(NVT and GCMC). To obtain for instance an adsorption isotherm you are going to run GCMC simulations 
of the system adsorbent (solid) + adsorbate (fluid) in a given temperature, chemical potential and volume. 
The volume is related to the geometry and size of the solid, which is kept constant. 
Currently, Cassandra supports all kind of geometries, including triclinic. 
In  order to illustrate how to run GCMC simulation involving a rigid solid we are presenting an example 
for adsorption of methane in Silicalite (MFI) at 300K. 
A very detailed information about this example is available in Examples/GCMC/CH4\_in\_Silicalite/README

\section{Files required} \label{sec:solid_files}
In order to run a simulation involving rigid solids and surfaces you need to provide the following files:

\begin{itemize}
\item A molecular connectivity file (MCF) (*.mcf) containing the molecular connectivity 
information for the solid and the fluid
\item A file with the coordinates of the all the atoms of solid or surface (*.xyz) 
in order to be able to start a simulation from a empty solid.
\item A fragment library file with fragment information 
\item An input file (*.inp)   
\end{itemize} 

The MCF and XYZ files can be obtained starting from a PDB file.
 
In this session we are going to present just the generation of the files for the solids and surfaces. 

\subsection{Getting a PDB file}\label{sec:solid_pdb}
There are many ways to obtain a PDB file for a solid. For instance, a zeolite type Silicalite you can get from:

\begin{itemize}
\item Literature:
D. H. Olson, G. T. Kokotailo, S. L. Lawton and W. M. Meier
Crystal Structure and Structure-Related Properties of ZSM-5
The Journal of Physical Chemistry, Vol. 85, No. 15, 1981

\item Starting from a CIF file:
The CIF file is provided in the Database of Zeolite Structure (http://www.iza-structure.org/databases/).
Enter in the page then click on the "Framework Type" menu and select your zeolite. 
The open window will show you many information about the zeolite type and will give 
access to the CIF file. The CIF file contains information about the zeolite structure 
such as cell parameters, space groups, T and O atoms coordinates. 
You can  convert a CIF file in a PDB file using the free version of both software: MERCURY and VESTA.
Example using VESTA:

Step 1:
In the file menu:
click Open... and then download our CIF file (ex: MFI.cif)

Step 2:
In the Objects menu:
click Boundary... and then fill the Ranges for the fractional coordinates according the size you want.
Examples:
for a unit cell the ranges in x, y and z should be -0 to 1. The unit cell of MFI contains 288 atoms.
for a supercell 2x2x2 the ranges in x, y and z should be -1 to 1. 
This supercell will contain 2304 atoms = 8x288 atoms

Step 3:
In the file menu:
click Export Data... and then choose the name of the file with pdb extension (Ex: MFI8uc.pdb for a structure of MFI with 8 unit cells)

\end{itemize}

Done... You created a PDB file.


\subsection{Getting a MCF file}\label{sec:solid_mcf}
You are going to need the PDB file and information about the solid force field you are going to use in the simulation.
In literature, you can observe two approaches in order to get the solid force field. 
In one of them the values for the interaction parameters for each atom of the solid were 
tuning for a given mixing rule in order to best fit a set of data. 
Some authors also uses to set the interaction parameters for the solid equal to zero and 
then tuning the cross interactions parameters in order to best fit a set of data.

In the case of MFI there are a few force field available in literature

The way to convert the PDB file in a MCF file for a solid or surface 
follows the same procedure described for components in a fluid phase. See \ref{utility:mcfgen}

\subsection{Getting XYZ file}\label{sec:solid_xyz}

Once again there is many ways to convert the PDB file in a XYZ file. 
You can always do your own script or you can use free software like VMD.

Reference of VMD:
Humphrey, W., Dalke, A. and Schulten, K.
VMD - Visual Molecular Dynamics
J. Molec. Graphics 1996, 14.1, 33-38

\texttt{
> vmd "name".pdb \\
> set all [atomselect top all] \\
> \$all writexyz "name".xyz \\
}
\\ \\

Example:

\texttt{
> vmd MFI8uc.pdb \\
> set all [atomselect top all] \\
> \$all writexyz MFI8uc.xyz \\
}
\\ \\
Since the XYZ file created by VMD informs the number of atoms and Cassandra needs a 
file with the number of molecules of each species you will need to modify the xyz file generated by VMD. 
For detailed information about to convert that file in the one that  will be used for starting a simulation 
from an empty solid using the option Read\_Old see /Chapter4/Start/\_Type  T

Done. You have created a XYZ file in a proper format to be used in Cassandra.

\subsection{Preparing the input file}

At this point, you should have already prepared the MCF and XYZ files following the procedures described above. \\ 

Now you are going to prepare the input file for running a short simulation in order to generate the 
Fragment files for both components (the solid and the fluid). However, 
you will come across with the question: What are the proper values should 
I put in \texttt{Chemical\_Potential\_Info}?. This keyword "Chemical\_Potential\_Info" sets 
the chemical potential of the insertable species in the order in which they 
appear in the \texttt{Molecule\_Files} section. \\ 

The chemical potentials will be set arbitrarily to zero for species that cannot 
be swapped or exchanged with a reservoir. Note that the de Broglie wavelength of the 
species is automatically calculated and used in the acceptance rules (see \ref{sec:MuVT}). \\

Units of chemical potential are kJ/mol. Then, since you are not going to insert/remove any 
solid, you just need to inform the chemical potential for the other species 
(in this example: methane). Then, you will need to run an independent GCMC 
simulation for the fluid phase in order to get the appropriated value for chemical 
potential (See example in /Examples/GCMC/Methane/README). \\

Other important question that you can come across is about the move 
probabilities \texttt{Move\_Probability\_Info}. 
Remember that the solid does not have freedom of move. 
Then all the probabilities involving the solid must be set equal to zero. \\

Now you are ready to obtain the Fragment\_Files. \\

\subsection{Getting the Fragment Files}\label{sec:fragment file}
Since the solid or surface is considered to be rigid there is just one possible conformation 
for the atoms. You can use the script provided in \ref{utility:libgen} to
automatically generate the files for the \texttt{Fragment\_Files}.
The script will take information from the PDB file. \\


Done. You are ready for running your simulation.
