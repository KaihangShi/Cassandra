\chapter{Features}

\section{Slit\_Pore} 
%
This section describes the changes to the code to enable a simulation
in a slit pore geometry. The slit pore is defined to be the geometry
in which there is a 2-D periodicity and finite length in the third 
dimension. Cassandra handles the 2-D periodicity in the X and Y directions;
the finite length is aligned along the Z direction. In the input file
to the Cassandra executable, the following indicate that simulation
involves a slit pore geometry: \\ \\
%
{\bf input\_routines.f90} \\ \\
%
The box shape is defined to be ``Slit\_Pore'' and the lengths in X and
Y directions are read in along with the pore width in the Z direction.
The pore width is stored in the variable ``{\em pore\_width}''. For the ease
of calculation of various properties related to the box, the pore width
is set as the dimension in the z direction. Note that
this value does not have any impact on the actual simulation. 
Whether a given simulation box has a slit pore geometry is specified by
the logical array variable `{\em l\_slit\_pore}'\\ \\
%
For the solid framework calculations, another routine, 'Get\_Solid\_Potential\_Type'
has been added. This routine determines the type of solid potential that is to
be used in the simulation. The keyword is specified by '\# Solid\_Potential\_Type'.
At present, acceptable value is 'Steele'. More potential types can be easily added. \\ \\
%
{\bf minimum\_image\_separation.f90} \\ \\
%
All the routines `Minimum\_Image\_Separation', 'Apply\_PBC\_Int' and
'Fold\_Molecule' apply PBC in two dimensions if the variable '{\em
l\_slit\_pore}' is evaluated to be TRUE. \\ \\
%
A new routine 'Compute\_Molecule\_Position\_Slit\_Pore' has also been
added to this file. The purpose of this routine is to evaluate the
z coordinates of all the atoms of the molecule undergoing perturbation
to enusre that they are confined to the slit pore. If any of the atoms
crosses the pore wall boundaries then an overlap flag is set to .TRUE.
and the corresponding move will be rejected.
%
{\bf Solid Fluid Interactions for a Structureless Wall} \\ \\
%
The interactions due to a structureless wall on a fluid is 
represented by the Steele 10-4-3 potential that has the form
%
\begin{equation}
\phi_{sf} (z)  =  2 \pi \rho_s \epsilon_{sf} \sigma_{sf}^2 \Delta
 \left [ \frac {2} {5} \left ( \frac {\sigma_{sf}}{z} \right )^{10}
  - \left ( \frac {\sigma_{sf}} {z} \right )^4
  - \left ( \frac {\sigma_{sf}^4} {3\Delta(z+0.61*\Delta)^3} \right ) \right ].
\label{Eq:Steele_Pot}
\end{equation}
where the $\epsilon$ and $\sigma$ have their usual meaning and are the cross
interaction parameters between the wall and fluid of interest. The potential
parameters and other parameters in the above equation correpsond to those
representative of a graphite wall \\ \\
%
$\sigma_{ss} = 0.340$ nm , $\epsilon_{ss}/K = 28.0$ K \\ \\
%
$\rho_s = 114$ nm$^{-3}$, $\Delta = 0.335$ nm. \\ \\
%
The cross interaction parameters are calculated using the Lorentz-Berthelot
combining rule. In the code, the length parameters are input in \AA while
the energy paramter in the atomic units and conversions are carried out 
accordingly. \\ \\
%
{\bf create\_nonbond\_table.f90} \\ \\
%
Since the prefactor in equation~\ref{Eq:Steele_Pot} is a constant during
the course of a simulation for each atomtype, it is calculated and
stored in the array variable '{\em pre\_steele}'. Similarly, the cross
parameter $\sigma_{sf}$ is precalculated and stored in the array 
variable '{\em sigma\_steele}'. Additional variables are defined as
'{\em sigma\_ss}', '{eps\_ss}', '{rho\_s}', 'delta\_s'. Also, two more
variables corresponding to the the quanties that remain constant during
the simulations are defined: \\ \\
%
sig\_delta(:) = sigma\_steele(:) \/ delta\_s \/ 3000.0 \\ \\
%
The factor $3000.0$ is due to the conversion of nm$^{-3}$ to $\AA^3$. 
Another constant is \\ \\
%
delta\_constant = 0.61 * delta\_s \\ \\
%
For a given pore width H, the contibution of the pore walls to the
total potential at any point $z$ in the pore is given by 
\begin{equation}
\phi_{pore}(z) = \phi_{sf}(z) + \phi_{sf}(H-z).
\label{Eq:Total_Wall_Potential}
\end{equation}
%
{\bf energy\_routines.f90} \\ \\
%
This subroutine is modified by adding a routine that calculates the 
potential energy of the molecule in the presence of Steele potential
wall. The new subroutine is labled as \\ \\
%
Compute\_Molecule\_Framework\_Energy. \\ \\
%
Though at present the routine computes potential energy due to
walls only, additional wall potential and look up tables for the
zeolitic systems will be placed here. The total energy due to the framework
will be returned as {\em E\_framework}. {\bf Note that currently there isn't
 a routine to calculate the energy of a single atom or a fragment
with the framework. Something that needs to be done in the future.} \\ \\
%
For electrostatic systems, a 3D Ewald summation has been implemented
with a correction term as given in the paper by Yeh and Berkowitz. The
correction term depends on the dipole of the simulation box in z
direction and is given by
% Correction term for 3D Ewald summation in slab geometry
%
\begin{equation}
E_{correction} = \frac {2\pi} {V} \left ( \sum_{i = 1}^{i=N} q_i z_i \right )^2
\end{equation}
The correction energy is stored in the 'energy' class by the variable
'ewald\_correction'. Additionally, a 'dipole\_z' vector was defined to
hold the z component of the dipole of a given box. When this routine
is called to computed the Ewald correction term, 'dipole\_z' gets
reset. Hence if a move is rejected, the dipole needs to be switched
back to the original value. \\ \\
%
{\bf translate.f90} \\ \\
As with any other energy calculations, the energy of the molecule with
the framework is calculated before. The translation is performed in all
the directions. However, if the simulation involves slit pore geometry
then a check is performed to ensure that the entire molecule is still
within the pore after the translation has been performed. If this condition
is violated the move is immediately rejectd. In future, the energy of
the molecule with framework will be stored so that the computation
of the old energy can be avoided. \\ \\
%
{\bf insertion.f90} \\ \\
%
The insertion is to be carried out within the pore volume. So the random
position of the molecule is generated along the X and Y directions in a
standard way while that in the z direction is confined between 0 and the
width of the pore. i.e.,
\begin{equation}
{\displaystyle molecule\_list(alive,is)\%zcom = rranf() * pore\_width}
\end{equation}


If an atom crosses the pore boundaries, the move is
immediately rejected as in the translate move. Framework energy is calculated
if all the atomic positions fall within the pore volume and this energy
is added to the change in energy required in the final acceptance rule.






