\chapter{Introduction}
Cassandra is an open source Monte Carlo package capable of simulating
any number of molecules composed of rings, chains, or both. It can be
used to simulate compounds such as small organic molecules, oligomers,
aqueous solutions and ionic liquids. It handles a standard ``Class
I''-type force field having fixed bond lengths, harmonic bond angles
and improper angles, a CHARMM or OPLS-style dihedral potential, a

Lennard-Jones 12-6 or Mie potential and fixed partial charges. It does {\em
  not} treat flexible bond lengths. Cassandra uses OpenMP parallelization and comes

with a number of scripts, utilities and examples to help with
simulation setup. \\ \\ 
%
Cassandra is capable of simulating systems in the following ensembles: \\ 
%
\begin{itemize}
\item Canonical (NVT) 
\item Isothermal-Isobaric (NPT) 
\item Grand canonical ($\mu$VT) 
\item Constant volume Gibbs (NVT-Gibbs) 
\item Constant Pressure Gibbs (NPT- Gibbs)
\end{itemize}

\section{Distribution}
Cassandra is distributed through \texttt{conda-forge} and a source code tarball. Installation
through \href{https://docs.conda.io/en/latest/miniconda.html}{conda} is easiest: \\ \\
%
\texttt{> conda create --name cassandra -c conda-forge cassandra} \\
%

The previous command creates a new conda environment (named ``cassandra''),
and installs \texttt{cassandra} from the \texttt{conda-forge} channel channel.
You can verify that the installation completed by activating the environment: \\ \\
%
\texttt{> conda activate cassandra} \\ \\
%
and then checking that \texttt{cassandra.exe} is on your \texttt{PATH}: \\ \\
%
\texttt{> which cassandra.exe} \\ \\
%
should find the executable. \texttt{mcfgen.py} and \texttt{library\_setup.py}
will also be on your \texttt{PATH} when the \texttt{cassandra} conda environment
is activated. If you choose to instead install from source,
you can unpack the distribution by running the command \\ \\
%
\texttt{> tar -xzf Cassandra\_V1.2.tar.gz} \\ \\
%
Upon successful unpacking of the archive file, the Cassandra\_V1.2 directory will have a number of subdirectories. Please refer to the README file in the main Cassandra directory for a detailed information on each of the subdirectories. \\ \\
%
\texttt{Documentation} - contains this user guide\\ \\
%
\texttt{Examples} - contains example input files and short simulations of various systems in the above ensembles. \\ \\
%
\texttt{MCF} - molecular connectivity files for a number of
molecules. These can be used as the basis for generating your own MCF files for
molecules of interest. \\ \\ 
%
\texttt{Scripts} - useful scripts to set up simulation input files. \\ \\
%
\texttt{Src} - Cassandra source code. \\ \\
