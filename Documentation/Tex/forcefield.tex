\chapter{Force Field} \label{Chapter:Force Field}


\section{Bonds} \label{Sec:Bonds}
Cassandra is designed assuming all bond lengths are fixed. If you wish
to utilize a force field developed with flexible bond lengths, we
recommend that you either use the nominal or ``equilibrium'' bond
lengths of the force field as the fixed bond lengths specified for a
Cassandra simulation or carry out an energy minimization of the
molecule with a package that treats flexible bond lengths and utilize
the bond lengths obtained from the minimization. 

\begin{center}
\begin{table}[h]
	\begin{center}
	\caption{Cassandra units for bonds}
	\begin{tabular} {l l l} \\ \hline \hline
	Parameter  & Symbol  & Units \\ \hline \\
	Bond length  & l  & \AA \\ \hline \\
	\end{tabular}
	\end{center}
	\label{Tab:Bond_Units}
\end{table}
\end{center}

\section{Angles}\label{Sec:Angles}

Cassandra supports two types of bond angles:
\begin{itemize}
\item `fixed' : The angle declared as fixed is not perturbed during the course of the simulation.
\item `harmonic' : The bond angle energy is calculated as 
\begin{equation}
E_\theta = K_\theta (\theta - \theta_0)^2
\label{Eq:angle_potential}
\end{equation}
where the user must specify $K_\theta$ and $\theta_0$. Note that a factor
of $1/2$ is {\bf not used} in the energy calculation of a bond
angle. Make sure you know how the force constant is defined in any
force field you use. 
\end{itemize}

\begin{center}
\begin{table}[h]
	\begin{center}
	\caption{Cassandra units for angles}
	\begin{tabular} {l l l} \\ \hline \hline
	Parameter  & Symbol  & Units \\ \hline \\
	Nominal bond angle & $\theta_0$ & degrees \\ 
	Bond angle force constant & $K_\theta$ & K/rad$^2$ \\ \\ \hline \\
	\end{tabular}
	\end{center}
	\label{Tab:Angle_Units}
\end{table}
\end{center}

\section{Dihedrals}\label{Sec:Dih}

Cassandra can handle four different types of dihedral angles:

\begin{itemize}

\item `OPLS': The functional form of the dihedral potential is
\begin{equation}
E_\phi = a_0 + a_1\, \left ( 1 + \cos(\phi)  \right ) + a_2 \, \left ( 1 -
  \cos(2\phi)\right ) + a_3 \, \left ( 1 + \cos (3\phi)\right )
\label{Eq:phi_OPLS}
\end{equation}
where $a_0$, $a_1$, $a_2$ and $a_3$ are specified by the user.

\item `CHARMM': The functional form of the potential is
\begin{equation}
E_\phi = a_0  (1 + \cos (a_1\phi - \delta))
\label{Eq:phi_CHARMM}
\end{equation}
where $a_0$, $a_1$ and $\delta$ are specified by the user.

\item `harmonic': The dihedral potential is of the form:
\begin{equation}
E_\phi = K_\phi (\phi - \phi_0)^2
\label{Eq:phi_harmonic}
\end{equation}
where $K_\phi$ and $\phi_0$ are specified by the user. 

\item `none' : There is no dihedral potential between the given atoms.  

\end{itemize}

\begin{center}
\begin{table}[h]
	\begin{center}
	\caption{Cassandra units for dihedrals}
	\begin{tabular} {l l l} \\ \hline \hline
	Functional Form & Parameter & Units \\ \\
	OPLS & $a_0$, $a_1$,  $a_2$, $a_3$ & kJ/mol \\ \\
	CHARMM & $a_0$ & kJ/mol \\
	                 & $a_1$ & dimensionless \\
	                 & $\delta$ & degrees \\ \\
	 harmonic & $K_\phi$ & K/rad$^2$ \\
	                 & $\phi_0$ & degrees \\  \\\hline \\
	\end{tabular}
	\end{center}
	\label{Tab:Dihedral_Units}
\end{table}
\end{center}

\section{Impropers}\label{Sec:Imp}

Improper energy calculations can be carried out with the following two options:

\begin{itemize}

\item `none': The improper energy is set to zero for the improper angle.

\item `harmonic': The following functional form is used to calculate
  the energy due to an improper angle
%
\begin{equation}
E_\psi = K_\psi \left ( \psi - \psi_0 \right )^2
\end{equation}
\end{itemize}
where $K_\psi$ and $\psi_0$ are specified by the user.

\begin{center}
\begin{table}[h]
	\begin{center}
	\caption{Cassandra units for impropers}
	\begin{tabular} {l l l} \\ \hline \hline
	 Parameter      & Symbol   &  Units \\
	 Force constant & $K_\psi$ &  K/rad$^2$\\
	 Improper	& $\psi_0$ & degrees \\ \\ \hline \\
	
	\end{tabular}
	\end{center}
	\label{Tab:Improper_Units}
\end{table}
\end{center}

\section{Nonbonded}\label{Sec:NB}
The nonbonded interactions between two atoms $i$ and $j$ are due to repulsion-dispersion interactions and electrostatic interactions (if any). 
\subsection{Repulsion-Dispersion Interactions} \label{Sec:LJ}
The repulsion-dispersion interactions can take one of the following forms:

\begin{itemize}

\item Lennard-Jones 12-6 potential (LJ):

\begin{equation}
 {\cal V}(r_{ij})= 4 \epsilon_{ij} \left [  \left ( \frac {\sigma_{ij}} { r_{ij} }\right )^{12} - \left ( \frac {\sigma_{ij}} { r_{ij} }\right )^{6}\ \right ]
\end{equation}
where $\epsilon_{ij}$ and $\sigma_{ij}$ are the energy and size
parameters set by the user. For unlike interactions, different
combining rules can be used, as described elsewhere. Note that this
option only evaluates the energy up to a specified cutoff
distance.  As described below, analytic tail corrections to the pressure and energy can be specified to account for the finite cutoff distance. 

%\item CHARMM:

%\begin{equation}
%{\cal V} (r_{ij})=  \epsilon_{ij} \left [  \left ( \frac {r_{min,ij}} { r_{ij} }\right )^{12} - \left ( \frac {r_{min,ij}} { r_{ij} }\right )^{6}\ \right ]
%\end{equation}

\item Cut and shift potential:

\begin{equation}
{\cal V}(r_{ij})= 4 \epsilon_{ij} \left [  \left ( \frac {\sigma_{ij}} { r_{ij} }\right )^{12} - \left ( \frac {\sigma_{ij}} { r_{ij} }\right )^{6}\ \right ] -  4 \epsilon_{ij} \left [  \left ( \frac {\sigma_{ij}} { r_{cut}}\right )^{12} - \left ( \frac {\sigma_{ij}} { r_{cut} }\right )^{6}\ \right ]
\label{Eq:cut_shift}
\end{equation}
where $\epsilon_{ij}$ and $\sigma_{ij}$ are the energy and size
parameters set by the user and $r_{cut}$ is the cutoff distance. This
option forces the potential energy to be zero at the cutoff
distance. For unlike interactions, different 
combining rules can be used, as described elsewhere.

%%%%%%%%%%%%%%
% Cut and switch potential
%%%%%%%%%%%%%%%

\item Cut and switch potential:

\begin{equation}
 {\cal V}(r_{ij})= 4 \epsilon_{ij} \left [  \left ( \frac {\sigma_{ij}} { r_{ij} }\right )^{12} - \left ( \frac {\sigma_{ij}} { r_{ij} }\right )^{6}\ \right ] f
 \label{Eq:cut_switch}
\end{equation}
%
The factor $f$ takes the following values:

\begin{eqnarray}
	f = 
	\begin{cases}
	
		1.0 \, \, \, &  r_ {ij}  \le r_{on} \\
		\frac { (r_{off}^2 - r_{ij}^2) (r_{off}^2 - r_{on}^2 + 2r_{ij}^2)} {\left ( r_{off}^2 - r_{on}^2 \right )^3}  \, \, \,  & r_{on} \textless r_{ij} \textless r_{off}\\
		0.0 \, \, \, & r_{ij} \ge r_{off} 
		
	\end{cases}
\end{eqnarray}
where $\epsilon_{ij}$ and $\sigma_{ij}$ are the energy and size
parameters set by the user. This option smoothly forces the potential
to go to zero at a distance $r_{off}$, and begins altering the
potential at a distance of $r_{on}$. Both of these parameters must be
specified by the user. For unlike interactions, different 
combining rules can be used, as described elsewhere.


%%%%%%%%%%%%%%
% Mie potential
%%%%%%%%%%%%%%%

\item Mie potential (generalized form of LJ):

\begin{equation}
 {\cal V}(r_{ij})=  \left ( \frac{n}{n-m} \right ) \left ( \frac {n}{m} \right )^{\frac{m}{n-m}}\epsilon_{ij} \left [  \left ( \frac {\sigma_{ij}} { r_{ij} }\right )^{n} - \left ( \frac {\sigma_{ij}} { r_{ij} }\right )^{m}\ \right  ] 
 \label{Eq:mie}
\end{equation}



where $\epsilon_{ij}$ and $\sigma_{ij}$ are the energy and size
parameters and $n$ and $m$ are the repulsive and attractive exponents set by the user. This option allows for the use of a generalized LJ potential (for LJ, $n$ = 12 and $m$ = 6).  Note that this
option only evaluates the energy up to a specified cutoff
distance. Both n and m can take on separate integer or float values set by the user. For unlike interactions, different 
combining rules can be used, as described elsewhere.

\item Mie cut and shift potential:

\begin{equation}
 {\cal V}(r_{ij})=  \left ( \frac{n}{n-m} \right ) \left ( \frac {n}{m} \right )^{\frac{m}{n-m}}\epsilon_{ij} \left [  \left ( \frac {\sigma_{ij}} { r_{ij} }\right )^{n} - \left ( \frac {\sigma_{ij}} { r_{ij} }\right )^{m}\ \right  ] -  \left ( \frac{n}{n-m} \right ) \left ( \frac {n}{m} \right )^{\frac{m}{n-m}}\epsilon_{ij} \left [  \left ( \frac {\sigma_{ij}} { r_{cut}}\right )^{n} - \left ( \frac {\sigma_{ij}} { r_{cut} }\right )^{m}\ \right]  
 \label{Eq:mie_cut_shift}
\end{equation}



where $\epsilon_{ij}$ and $\sigma_{ij}$ are the energy and size
parameters and $n$ and $m$ are the repulsive and attractive exponents set by the user. This
option forces the potential energy to be zero at the cutoff
distance (i.e. setting $n$ = 12 and $m$ = 6 provides the same potential as the LJ cut and shift option). For unlike interactions, different 
combining rules can be used, as described elsewhere.


\item Tail corrections: If the Lennard-Jones potential is used, standard Lennard-Jones tail corrections are used to approximate the long range dispersion interactions

\end{itemize}

\begin{center}
\begin{table}[h]
	\begin{center}
	\caption{Cassandra units for repulsion-dispersion interactions}
	\begin{tabular} {l l l} \\ \hline \hline
	 Parameter & Symbol &  Units \\
	Energy parameter 	& $\epsilon/k_B$ & K \\
	Collision diameter &	 $\sigma$ & \AA \\ \\ \hline \\
	\end{tabular}
	\end{center}
	\label{Tab:LJ_Units}
\end{table}
\end{center}

\subsection{Electrostatics}\label{Sec:qq}

Electrostatic interactions are given by  Coulomb's law

\begin{equation}
{\cal V}_{elec} (r_{ij}) = \frac{1}{4\pi\epsilon_0} \frac {q_i q_j} {r_{ij}}.
\label{Eq:Coulomb}
\end{equation}
where $q_i$ and $q_j$ are partial charges specified by the user and
placed on atomic positions given by $r_i$ and $r_j$. In a simulation,
the electrostatic interactions are calculated using either an Ewald
summation or a direct summation using the minimum image convention. Note that 
the total energy that is printed out in the property file is extensive. 
Consequently, to obtain intensive energies, the printed energies must divided by 
the total number of molecules in the system. 

\begin{center}
\begin{table}[h]
	\begin{center}
	\caption{Cassandra units for coulombic interactions}
	\begin{tabular} {l l l} \\ \hline \hline
	 Parameter & Symbol &  Units \\
	Charge &	 $q$ & e \\ \\ \hline \\
	\end{tabular}
	\end{center}
	\label{Tab:LJ_Units}
\end{table}
\end{center}

%\section{Cassandra Units}
%The following table provides a summary of the units used in Cassandra:

\begin{center}
\begin{table}[h]
	\begin{center}
	\caption{Summary of Cassandra units for input variables}
	\begin{tabular} {l l l} \\ \hline \hline
	Bond length  & l  & \AA \\ \hline \\
	\multicolumn{3}{c} {\bf Angles} \\ \\
	Nominal bond angle & $\theta_0$ & degrees \\ 
	Bond angle force constant & $K_\theta$ & K/rad$^2$ \\ \\ \hline \\
	\multicolumn{3}{c} {\bf{Dihedral angle}}  \\  \\
	OPLS & $a_0$, $a_1$,  $a_2$, $a_3$ & kJ/mol \\ \\
	CHARMM & $a_0$ & kJ/mol \\
	                 & $a_1$ & dimensionless \\
	                 & $\delta$ & degrees \\ \\
	 harmonic & $K_\phi$ & K/rad$^2$ \\
	                 & $\phi_0$ & degrees \\  \\\hline \\
	 \multicolumn{3}{c}{\bf {Improper angle}} \\ \\
	 Force constant & $K_\psi$ &  K/rad$^2$\\
	 Improper angle	& $\psi_0$ & degrees \\ \\ \hline \\
				\multicolumn{3}{c}{\bf Nonbonded} \\ \\
	Energy parameter 	& $\epsilon/k_B$ & K \\
	Collision diameter &	 $\sigma$ & \AA \\
	Charge			& q & e \\  \\ \hline \\
				\multicolumn{3}{c}{\bf {Simulation Parameters}} \\ \\
	Simulation box length & & \AA \\
	Volume                       & & \AA$^3$ \\
	Distances                   & & \AA \\
	Rotational width         & & degrees \\
	Temperature               & & K \\
	Pressure                      & & bar \\
	Chemical potential       & & kJ/mol \\
	Energy                   & & kJ/mol \\ \hline \hline
	
	\end{tabular}
	\end{center}
	\label{Tab:Cassandra_Units}
\end{table}
\end{center}

